\documentclass[
]{jss}

\usepackage[utf8]{inputenc}

\author{
Kristen Hunter\\Harvard University\\
Department of Statistics \And Luke Miratrix\\Harvard Graduate\\
School of Education \And Kristin Porter\\MDRC
}
\title{Power Under Multiplicity Project (\pkg{PUMP}): Estimating Power,
Minimum Detectable Effect Size, and Sample Size When Adjusting for
Multiple Outcomes}

\Plainauthor{Kristen Hunter, Luke Miratrix, Kristin Porter}
\Plaintitle{Power Under Multiplicity Project (\pkg{PUMP}): Estimating
Power, Minimum Detectable Effect Size, and Sample Size When Adjusting
for Multiple Outcomes}
\Shorttitle{\pkg{PUMP}: Power Under Multiplicity Project}


\Abstract{
For randomized controlled trials (RCTs) with a single intervention being
measured on multiple outcomes, researchers often apply a multiple
testing procedure (such as Bonferroni or Benjamini-Hochberg) to adjust
\(p\)-values. Such an adjustment reduces the likelihood of spurious
findings, but also changes the statistical power, sometimes
substantially, which reduces the probability of detecting effects when
they do exist. However, this consideration is frequently ignored in
typical power analyses, as existing tools do not easily accommodate the
use of multiple testing procedures. We introduce the \texttt{PUMP}
\texttt{R} package as a tool for analysts to estimate statistical power,
minimum detectable effect size, and sample size requirements for
multi-level RCTs with multiple outcomes. Multiple outcomes are accounted
for in two ways. First, power estimates from \texttt{PUMP} properly
account for the adjustment in \(p\)-values from applying a multiple
testing procedure. Second, as researchers change their focus from one
outcome to multiple outcomes, different definitions of statistical power
emerge. \texttt{PUMP} allows researchers to consider a variety of
definitions of power, as some may be more appropriate for the goals of
their study. The package estimates power for frequentist multi-level
mixed effects models, and supports a variety of commonly-used RCT
designs and models and multiple testing procedures. In addition to the
main functionality of estimating power, minimum detectable effect size,
and sample size requirements, the package allows the user to easily
explore sensitivity of these quantities to changes in underlying
assumptions.
}

\Keywords{power, multiple testing, multi-level models, randomized
controlled trials, \proglang{R}}
\Plainkeywords{power, multiple testing, multi-level models, randomized
controlled trials, R}

%% publication information
%% \Volume{50}
%% \Issue{9}
%% \Month{June}
%% \Year{2012}
%% \Submitdate{}
%% \Acceptdate{2012-06-04}

\Address{
        Kristin Porter\\
    Universitat Autònoma de Barcelona\\
    MDRC\\
475 14th Street\\
Suite 750\\
Oakland, CA 94612-1900\\
  
  
  }


% tightlist command for lists without linebreak
\providecommand{\tightlist}{%
  \setlength{\itemsep}{0pt}\setlength{\parskip}{0pt}}




\usepackage{amsmath}

\begin{document}



\section{Introduction}
\label{sec:intro}

The \texttt{PUMP} \texttt{R} package fills in an important gap in
open-source software tools to design multi-level randomized controlled
trials (RCTs) with adequate statistical power. With this package,
researchers can estimate statistical power, minimum detectable effect
size (MDES), and needed sample size for multi-level experimental
designs, in which units are nested within hierarchical structures such
as students nested within schools nested within school districts. The
statistical power is calculated for estimating the impact of a single
intervention on multiple outcomes. The package uses a frequentist
framework of mixed effects regression models, which is currently the
prevailing framework for estimating impacts from experiments in
education and other social policy research.\footnote{Other options
  include nonparametric or Bayesian methods, but these are less
  prevalent in applied research (for example, see
  \citet{GELMANETAL2012}, \citet{GelmanHill2007}).}

To our knowledge, none of the existing software tools for power
calculations allow researchers to account for multiple hypothesis tests
and the use of a multiple testing procedure (MTP). MTPs adjust
\(p\)-values to reduce the likelihood of spurious findings when
researchers are testing for effects on multiple outcomes. This
adjustment can result in a substantial change in statistical power,
greatly reducing the probability of detecting effects when they do
exist. Unfortunately, when designing studies, researchers who plan to
test for effects on multiple outcomes and employ MTPs frequently ignore
the power implications of the MTPs.

Also, as researchers change their focus from one outcome to multiple
outcomes, multiple definitions of statistical power emerge
(\citet{RN23882}; \citet{RN23878}; \citet{RN23881}; \citet{MTSAS}). The
\texttt{PUMP} package allows researchers to consider multiple
definitions of power, selecting those most suited to the goals of their
study. The definitions of power include:

\begin{itemize}
\tightlist
\item
  \textbf{individual power}: the probability of detecting an effect of a
  particular size (specified by the researcher) or larger for each
  hypothesis test. Individual power corresponds to how power is defined
  when there is focus on a single outcome.
\item
  \textbf{\(1-\)minimal power}: the probability of detecting effects of
  at least a particular size on at least one outcome. Similarly, the
  researcher can consider \textbf{\(d-\)minimal power} for any \(d\)
  less than the number of outcomes, or fractional powers, such as
  \(1/2-\)minimal power.
\item
  \textbf{complete power}: the power to detect effects of at least a
  particular size on \emph{all} outcomes.
\end{itemize}

As noted in \citet{Porter2018}, the prevailing default in many
studies---individual power---may or may not be the most appropriate type
of power. If the researcher's goal is to find statistically significant
estimates of effects on most or all primary outcomes of interest, then
their power may be much lower than anticipated when multiplicity
adjustments are taken into account. On the other hand, if the
researcher's goal is to find statistically significant estimates of
effects on at least one or a small proportion of outcomes, their power
may be much better than anticipated. In both of these cases, by not
accounting for both the challenges and opportunities arising from
multiple outcomes, a researcher may find they have wasted resources,
either by designing an underpowered study that cannot detect the desired
effect sizes, or by designing an overpowered study that had a larger
sample size than necessary. We introduce the \texttt{PUMP} package to
allow for directly answering questions that take multiple outcomes into
account, such as:

\begin{itemize}
\tightlist
\item
  How many schools would I need to detect a given effect on at least
  three of my five outcomes?
\item
  What size effect can I reliably detect on each outcome, given a
  planned MTP across all my outcomes?
\item
  How would the power to detect a given effect change if only half my
  outcomes truly had impact?
\end{itemize}

The methods in the PUMP package build on those introduced in
\citet{Porter2018}. This earlier paper focused only on a single RCT
design and model --- a multisite RCT with the blocked randomization of
individuals, in which effects are estimated using a model with
block-specific intercepts and with the assumption of constant effects
across all units. This earlier paper also did not produce software to
assist researchers in implementing its methods. With this current paper
and with the introduction of the PUMP package, we extend the methodology
to nine additional multi-level RCT designs and models. Also, while
\citet{Porter2018} focused on estimates of power, PUMP goes further to
also estimate MDES and sample size requirements that take multiplicity
adjustments into account.

\texttt{PUMP} extends functionality of the popular PowerUp! \texttt{R}
package (and its related tools in the form of a spreadsheet and Shiny
application), which compute power or MDES for multi-level RCTs with a
single outcome (\citet{RN4473}). For a wide variety of RCT designs with
a single outcome, researchers can take advantage of closed-form
solutions and numerous power estimation tools. For example, in education
and social policy research, see \citet{RN4473}; \citet{RN30153};
\citet{RN23884}; \citet{RN24179}. However, closed-form solutions are
difficult or impossible to derive when a MTP is applied to a setting
with multiple outcomes. Instead, we use a simulation-based approach to
achieve estimates of power.

In order to calculate power, the researcher specifies information about
the sample size at each level, the minimum detectable effect size for
each outcome, the level of statistical significance, and parameters of
the data generating distribution. The minimum detectable effect size is
the smallest true effect size the study can detect with the desired
statistical significance level, in units of standard deviations. An
``effect size'' generally refers to the standardized mean difference
effect size, which ``equals the difference in mean outcomes for the
treatment group and control group, divided by the standard deviation of
outcomes across subjects within experimental groups'' (\citet{RN27978}).
Researchers often use effect sizes to standardize outcomes so that
outcomes with different scales can be directly compared.

The package includes three core functions:

\begin{itemize}
\tightlist
\item
  \texttt{pump\_power()} for calculating power given a experimental
  design and assumed model, parameters, and minimum detectable effect
  size.
\item
  \texttt{pump\_mdes()} for calculating minimum detectable effect size
  given a target power and sample sizes.
\item
  \texttt{pump\_sample()} for calculating the required sample size for
  achieving a given target power for a given minimum detectable effect
  size.
\end{itemize}

For any of these core functions, the user begins with two main choices.
First, the user chooses the assumed design and model of the RCT. The
\texttt{PUMP} package covers a range of multi-level designs, up to three
levels of hierarchy, that researchers typically use in practice, in
which research units are nested in hierarchical groups. Our power
calculations assume the user will be analyzing these RCTs using
frequentist mixed-effects regression models, containing a combination of
fixed or random intercepts and treatment impacts at different levels, as
we explain in detail in Section\textasciitilde{}\ref{sec:est_power} and
in the Technical Appendix. Second, the user chooses the MTP to be
applied. \texttt{PUMP} supports five common MTPs --- Bonferroni, Holm,
single-step and step-down versions of Westfall-Young, and
Benjamini-Hochberg. After these two main choices, the user must also
make a variety of decisions about parameters of the data generating
distribution.

The package also includes functions that allow users to easily explore
power over a range of possible values of parameters. This exploration
encourages the user to determine the sensitivity of estimates to
different assumptions. \texttt{PUMP} also visually displays results.
These additional functions include:

\begin{itemize}
\tightlist
\item
  \texttt{pump\_power\_grid()}, \texttt{pump\_mdes\_grid()}, and
  \texttt{pump\_sample\_grid()} for calculating the given output over a
  range of possible parameter values.
\item
  \texttt{update()} to re-run an existing calculation with a small
  number of parameters updated.
\item
  \texttt{plot()} on \texttt{PUMP}-generated objects to generate plots
  (including grid outputs).
\end{itemize}

The authors of the \texttt{PUMP} package have also created a web
application built with R Shiny. This web application calls the
\texttt{PUMP} package and allows users to conduct calculations with a
user-friendly interface, but it is less flexible than the package, with
a focus on simpler scenarios (e.g., 10 or fewer outcomes). The app can
be found at \url{https://mdrcpump.shinyapps.io/pump_shiny/}.

The remainder of this paper is organized as follows. In
Section\textasciitilde{}\ref{sec:diplomas}, we introduce Diplomas Now,
an educational experiment, to be used as a running example throughout
the paper. We note, however, that the problem of power estimation for
multi-level RCTs is not exclusive to the educational setting. In
Section\textasciitilde{}\ref{sec:mtp_overview}, we provide a summary of
the multiple testing problem. Also in
Section\textasciitilde{}\ref{sec:mtp_overview}, we present an overview
of the statistical challenges introduced by multiple hypothesis testing
and how MTPs protect against spurious impact findings. In
Section\textasciitilde{}\ref{sec:est}, we introduce our methodology for
estimating power when taking the use of MTPs into account. This section
also briefly discusses our validation process.
Section\textasciitilde{}\ref{sec:choices} discusses the various choices
a user must make when using the package, including the designs and
models, MTPs, and key design and model parameters.
Section\textasciitilde{}\ref{sec:vignette} provides a detailed
presentation of the \texttt{PUMP} package with multiple examples of
using the packages functions to conduct calculations for our education
RCT example. Section\textasciitilde{}\ref{sec:conclusion} is a brief
conclusion.

\hypertarget{code-formatting}{%
\subsection{Code formatting}\label{code-formatting}}

In general, don't use Markdown, but use the more precise LaTeX commands
instead:

\begin{itemize}
\item
  \proglang{Java}
\item
  \pkg{plyr}
\end{itemize}

One exception is inline code, which can be written inside a pair of
backticks (i.e., using the Markdown syntax).

If you want to use LaTeX commands in headers, you need to provide a
\texttt{short-title} attribute. You can also provide a custom identifier
if necessary. See the header of Section \ref{r-code} for example.

\section[R code]{\proglang{R} code}\label{r-code}

Can be inserted in regular R markdown blocks.

\begin{CodeChunk}
\begin{CodeInput}
R> x <- 1:10
R> x
\end{CodeInput}
\begin{CodeOutput}
 [1]  1  2  3  4  5  6  7  8  9 10
\end{CodeOutput}
\end{CodeChunk}

\subsection[Features specific to rticles]{Features specific to
\pkg{rticles}}\label{features-specific-to}

\begin{itemize}
\tightlist
\item
  Adding short titles to section headers is a feature specific to
  \pkg{rticles} (implemented via a Pandoc Lua filter). This feature is
  currently not supported by Pandoc and we will update this template if
  \href{https://github.com/jgm/pandoc/issues/4409}{it is officially
  supported in the future}.
\item
  Using the \texttt{\textbackslash{}AND} syntax in the \texttt{author}
  field to add authors on a new line. This is a specific to the
  \texttt{rticles::jss\_article} format.
\end{itemize}

\bibliography{refs}



\end{document}
